\documentclass[../../layout.tex]{subfiles}

\begin{document}
\begin{resumo}
\hspace*{3em}
A tecnologia na atualidade permite que pessoas se conectem umas às outras e entre dispositivos inteligentes, tal como celulares, computadores e controladores. Com o aumento dessa demanda, as pessoas começaram a utilizar mais e mais desses dispositivos não só em casa, mas também em mercados, shoppings e hospitais. É evidente como eles estão conectados entre si via internet e conseguem numa fração de segundo trocar informações. Portanto, pesquisas nesta área de dispositivos conectados são de grande valia para facilitar o dia a dia das pessoas e negócios. O uso destes dispositivos pode ir além de simplesmente acionar um relê ou ligar um led, podendo atingir até mesmo a análise de dados em massa (\emph{Big Data}) e a Inteligência Artificial. Este modelo  é chamado de IoT (Internet das Coisas) e se trata exatamente do nosso projeto, visando desenvolver um sistema que inclua facilidade para o usuário final ao utilizar dispositivos inteligentes conectados entre si, com uma interface de controle amigável sem a dependência de serviços de terceiros e sem que o usuário tenha conhecimento profundo em programação ou eletrônica. Os métodos utilizados para criação desta solução foram escolhidos com base na praticidade, eficiência e aplicabilidade das ferramentas e principalmente nos conhecimentos adquiridos ao longo do curso.
\vspace{\onelineskip}

\noindent
\textbf{Palavras-chave}: Internet das Coisas. Dispositivos conectados. Código Aberto.
\end{resumo}
\end{document}
