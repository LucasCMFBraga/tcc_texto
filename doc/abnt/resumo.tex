\documentclass[../../layout.tex]{subfiles}

\begin{document}
\begin{resumo}
\hspace*{3em}
A maior parte da energia gerada hoje provém de fontes não sustentáveis como combustíveis fosseis e fissão nuclear. Estas fontes são responsáveis de aumentar os efeitos da poluição, causar mudanças climáticas irreversíveis e, eventualmente, irão se esgotar completamente. Portanto, pesquisas na área de geração de energia renovável é parte integral no avanço do consumo de energia moderno para um nível mais sustentável, sendo que empresas e indústrias têm começado a considerar o potencial de geração desse tipo de energias, o que depende de vários fenômenos meteorológicos complexos. A tarefa de análise do potencial de geração não é uma questão trivial e requer um sistema especializado para obter resultados precisos e várias tecnologias modernas como IoT (Internet of Things) e Inteligência Artificial podem ser utilizadas para aumentar ainda mais a qualidade do produto. Este projeto visa desenvolver um sistema completo para esta aplicação incluindo aquisição de dados, armazenamento, processamento de informação e interface de usuário. Os métodos utilizados para a criação deste solução de ponta a ponta foram escolhidos pensando-se na utilização de tecnologias emergentes e na aplicabilidade dos conhecimentos adquiridos ao longo do curso.
\vspace{\onelineskip}

\noindent
\textbf{Palavras-chave}: Energia renovável. Análise de viabilidade. Dados meteorológicos.
\end{resumo}
\end{document}
