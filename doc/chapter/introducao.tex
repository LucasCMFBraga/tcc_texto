\documentclass[../../layout.tex]{subfiles}

\begin{document}
\chapter{Introdução}
\hspace*{3em}Com o desenvolvimento elevado da tecnologia, a quantidade de dispositivos conectados à internet hoje em dia é relativamente maior que nas décadas anteriores. Estipula-se que objeve um aumento de cinco vezes em dez anos\cite{conectdevicesnum}. O uso desses dispositivos são variados e muitos deles estão dentro de nosso alcance, seja em casa ou serviços que utilizamos. Houveram ações e investimentos para que o desenvolvimento dessa tecnologia fosse impulsionada dentro do Brasil, especificamente para a área de IoT voltada para telecomunicações\cite{iotinvest}. Nos dias atuais com tantos serviços que utilizamos dentro de casa, seja para acessar um site, ouvir música, assistir vídeo ou navegar no GPS, estamos utilizando um dispositivo conectado e por trás dele tem uma empresa que controla esse acesso e provém facilidade e informações para o usuário final. Porém, devido à isso, muitas empresas cobram por esses serviços.\par
Cada vez mais pessoas estão conectadas e inseridas na tecnologia, onde os conceitos básicos da tecnologia tem se tornado cada vez mais comum na sociedade, tornando possível  que pessoas com baixo nível de conhecimento técnico consigam compreender e desenvolver projetos simples. Nos últimos anos surgiram muitos interessados na área de IoT (Internet da coisas) devido às facilidades e praticidades oferecidas por essa tecnologia, no entanto os interessados têm encontrado grandes obstáculos para desenvolver seus projetos pessoais, com a falta de plataformas baratas e que demandam alto nível de conhecimento técnico.\apr
Os projetos IoTs em geral são considerados complexos pois envolvem diversas áreas da tecnologia por este motivo são tidos como sistemas heterogêneos pois envolvem conhecimentos em, programação de sistemas embarcados, análise de esquemas eletrônicos, redes de computadores, protocolos WEB e conhecimento em arquitetura WEB. \par
Porém esses fundamentos são essenciais para o desenvolvimento de uma simples aplicação IoT. Logo para permitir que pessoas com baixo conhecimento técnico tenham acesso a um simple ambiente de desenvolvimento IoT é necessário condensar o desenvolvimento desses recursos em uma única plataforma  que transcrever  de forma intuitiva as configurações essenciais para o usuário desenvolver uma mínima aplicação IoT.\cite{IoTeveryone} \par
Portanto a proposta desse projeto é  desenvolver uma plataforma IoT  genérica que permitam que pessoas de baixa e alto nível técnico possam desenvolver uma aplicação IoT de forma fácil e intuitiva.


\section{Objetivo}
\hspace*{3em}Este projeto têm seu valor implementado tanto em software quanto hardware para centralizar em um sistema com os seguintes objetivos:
\begin{enumerate}[label=\alph*)]
\itemsep0em
\item visualizar periféricos e dispositivos conectados ao sistema;
\item transmitir as informações dos dispositivos para o raspberry host;
\item controlar periféricos e dispositivos através da interface web;
\item proporcionar ao usuário a flexibilidade de programar as funcionalidades desta plataforma de forma fácil.
\end{enumerate}


\end{document}
