\documentclass[../../layout.tex]{subfiles}

\begin{document}
\chapter{Introdução}
\hspace*{3em}Com o desenvolvimento elevado da tecnologia, a quantidade de dispositivos conectados à internet hoje em dia é relativamente maior que nas décadas anteriores. Estipula-se que objeve um aumento de cinco vezes em dez anos\cite{conectdevicesnum}. O uso desses dispositivos são variados e muitos deles estão dentro de nosso alcance, seja em casa ou serviços que utilizamos. Houveram ações e investimentos para que o desenvolvimento dessa tecnologia fosse impulsionada dentro do Brasil, especificamente para a área de IoT voltada para telecomunicações\cite{iotinvest}. Nos dias atuais com tantos serviços que utilizamos dentro de casa, seja para acessar um site, ouvir música, assistir vídeo ou navegar no GPS, estamos utilizando um dispositivo conectado e por trás dele tem uma empresa que controla esse acesso e provém facilidade e informações para o usuário final. Porém, devido à isso, muitas empresas cobram por esses serviços, produtos ou até mesmo coletam dados pessoais. Com esses fatores, a necessidade de mais sistemas e empresas com foco em disponibilizar praticidade e conforto para o usuário final é certa.

\section{Objetivo}
\hspace*{3em}Este projeto têm seu valor implementado tanto em software quanto hardware para centralizar em um sistema com os seguintes objetivos:
\begin{enumerate}[label=\alph*)]
\itemsep0em
\item visualizar periféricos e dispositivos conectados ao sistema;
\item transmitir as informações dos dispositivos para o raspberry host;
\item controlar periféricos e dispositivos através da interface web;
\item proporcionar ao usuário a flexibilidade de programar as funcionalidades desta plataforma de forma fácil.
\end{enumerate}


\end{document}
